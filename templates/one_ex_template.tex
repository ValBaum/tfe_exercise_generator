\setcounter{section}{0}
\section{Pointeurs}

Soit l'état de la mémoire suivant:

\begin{table}[!htbp]
  \centering
  \begin{tabular}{ll|l|}
%(memory)s
  \end{tabular}
\end{table}

La première colonne donne le nom des variables, la deuxième donne des adresses
mémoires (nous les exprimerons dans ce challenge en \textbf{base 10}) et, enfin,
la troisième donne la valeur stockée à cette adresse. De plus, voici comment ont été déclarées les variables (on suppose que les entiers (\texttt{int}) sont représentés sur 4
octets\footnote{les \texttt{short} sur 2, les \texttt{long sur 8}}):

\begin{lstlisting}
%(declaration)s
\end{lstlisting}

Dans ce Challenge, vous devez indiquer la valeur des \textbf{expressions}
suivantes. Considérez qu'entre chaque expression, la mémoire est réinitialisée
telle que présentée dans le schéma. Si un accès mémoire est demandé à une
adresse hors de l'intervalle [%(minMemory)s, %(maxMemory)s], mentionnez l'erreur de segmentation par la valeur \texttt{SF}\footnote{Pour \texttt{S}egmentation \texttt{F}ault.}. Les adresses sont aussi représentées sur 4 octets. 

\newpage

\begin{enumerate}
%(exercises)s

\end{enumerate}
\newpage


